\subsection{Wyniki analizy częstotliwościowej}

Wyniki analizy pokazały znaczące różnice w rozkładzie częstotliwości dla badanych języków. Dominujące pasma częstotliwości dla angielskiego wynosiły 2-14 kHz, co jest zgodne z teorią Tomatisa. Natomiast dla niemieckiego dominowały pasma w zakresie 125-3500 Hz.

\subsection{Przykłady wyników}

W Tabeli \ref{tab:freq_ranges} przedstawiono zakresy częstotliwości dla poszczególnych języków.

\begin{table}[h]
    \centering
    \caption{Zakresy częstotliwości dla wybranych języków}
    \begin{tabular}{lccc}
        \toprule
        Język & Zakres częstotliwości (Hz) & Dominujące formanty & Średnia częstotliwość \\
        \midrule
        Angielski & 2000-14000 & 3 & 7000 \\
        Niemiecki & 125-3500 & 2 & 2000 \\
        \bottomrule
    \end{tabular}
    \label{tab:freq_ranges}
\end{table}
