Teoria dra Alfreda Tomatisa zakłada, że różne języki charakteryzują się specyficznymi pasmami częstotliwości dla poszczególnych głosek. Na przykład język angielski operuje w zakresie od 2 do 14 kHz, podczas gdy niemiecki w paśmie 125-3500 Hz. W niniejszej pracy sprawdzono, czy można potwierdzić te twierdzenia w oparciu o analizę mowy dla różnych języków.

W ramach projektu przeanalizowano nagrania mowy dla różnych języków, wykorzystując bazy danych oraz narzędzia akustyczne, takie jak librosa i praat-parselmouth. Celem projektu było wyznaczenie zakresów częstotliwości charakterystycznych dla poszczególnych języków oraz ich porównanie z teorią Tomatisa.
