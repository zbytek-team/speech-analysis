Na podstawie przeprowadzonej analizy można stwierdzić, że różne języki rzeczywiście charakteryzują się odmiennymi pasmami częstotliwości. Teoria dra Alfreda Tomatisa znalazła potwierdzenie w przypadku badanych języków (angielski, niemiecki). Przeprowadzona analiza mowy przy użyciu narzędzi librosa i praat-parselmouth pozwoliła na dokładną ekstrakcję cech akustycznych i ich porównanie między językami.
