Projekt ten ma na celu zbadanie różnic w pasmach częstotliwości mowy dla różnych języków oraz weryfikację teorii dra Alfreda Tomatisa, która zakłada, że różne języki mają charakterystyczne zakresy częstotliwości dla poszczególnych głosek. Przeprowadzona analiza opierała się na bazach danych mowy oraz narzędziach akustycznych takich jak librosa i praat-parselmouth.
